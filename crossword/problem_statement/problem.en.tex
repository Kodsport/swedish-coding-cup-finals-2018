\problemname{Crossword}
Two of the Coding Cup judges, Johan and Simon, are both fans of crosswords.
This recently culminated in them creating a \href{http://boi2018.progolymp.se/day4.pdf}{crossword of their own},
for the Baltic Olympiad.

Now for the Coding Cup, we ask you to do the same. There are a couple of rules a good crossword should obey:
\begin{itemize}
  \item All words should actually exist. To be precise, they should come from \href{http://codingcup.se/2018/wordlist.txt}{this} English-language wordlist.
  \item The grid should be connected.
  \item The density of the crossword should be as high as possible.
\end{itemize}

Given a number of rows and columns, construct a crossword with that size and as high a
density as you can, obeying the first two constraints above.
You will be scored on the density you obtain.
To help you out, we've already put in a couple of the letters.

\section*{Input}
The first line of input contains the test case number $T$ ($0 \le T \le 5$ -- $0$ denotes the sample group).
The second line of input contains the two integers $R, C$ ($1 \le R, C \le 500$).
Then follow $R$ lines with $C$ characters each. Each character is either \texttt{.},
meaning a blank space which you can fill in, or a letter \texttt{a-z}, one of the
letters we have already filled in for you.

The input files are open (and you're encouraged to solve them offline and simply
submit a program which prints your solution). They can be accessed through the
sidebar on the right-hand side of the problem page on Kattis.

\section*{Output}
Output $R$ lines with $C$ characters each, the filled-in crossword.
All characters should be either \texttt{a-z} or \texttt{.}, and all filled-in
letters must be preserved.

The grid of characters \texttt{a-z} must be connected, and every word
(separated by \texttt{.} or the boundaries of the crossword, in both
horizontal and vertical direction) must be legal according to the wordlist
linked above.

There is guaranteed to be a valid solution for each input.

\section*{Scoring}
Your solution will be tested on 5 test cases, each worth a maximum of 20 points.
Your score for each test case will be $20$ times the proportion of characters
of the grid that you have filled in with \texttt{a-z}.
For example, if the sample would have been a real test case, the sample solution
would score $19 / 80 \cdot 20 = 4.75$ points, since it filled in $19$ squares out of $80$.
