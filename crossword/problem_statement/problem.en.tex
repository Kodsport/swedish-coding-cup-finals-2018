\problemname{Crossword}
Two of the Coding Cup judges, Johan and Simon, are both fans of crosswords.
This recently culminated in them creating a \href{http://boi2018.progolymp.se/day4.pdf}{crossword of their own},
for the Baltic Olympiad.

Now for the Coding Cup, we ask you to do the same. There are a couple of rules a good crossword should obey:
\begin{itemize}
  \item All words should actually exist. To be precise, they should come from \href{http://codingcup.se/2018/wordlist.txt}{this English-language wordlist}.
  \item Every letter should be part of a word (i.e., there should be no single 1x1 squares).
  \item Ideally, the grid should be connected (i.e., it should be possible to move from any \texttt{a-z} square to any other,
    moving only in the four cardinal directions and without passing any empty squares).
  \item The density of the crossword should be as high as possible.
\end{itemize}

Given a number of rows and columns, construct a crossword with that size and as high a
density as you can, obeying the first two constraints above.
You will be scored on the density you obtain.
If your grid is not connected, your score will be lowered.

To help you out, we've already put in a couple of the letters.

\section*{Input}
The first line of input contains the test case number $T$ ($0 \le T \le 5$ -- $0$ denotes the sample group).
The second line of input contains the two integers $R, C$ ($1 \le R, C \le 1000$).
Then follow $R$ lines with $C$ characters each. Each character is either \texttt{.},
meaning a blank space which you can fill in, or a letter \texttt{a-z}, one of the
letters we have already filled in for you.

The input files are open (and you're encouraged to solve them offline and simply
submit a program which prints your solution). They can be accessed through the
sidebar on the right-hand side of the problem page on Kattis.

\section*{Output}
Output $R$ lines with $C$ characters each, the filled-in crossword.
All characters should be either \texttt{a-z} or \texttt{.}, and all filled-in
letters must be preserved.

Each square with \texttt{a-z} must be part of a word in some or both directions,
and every word (separated by \texttt{.} or the boundaries of the crossword, in both
horizontal and vertical direction) must be legal according to the wordlist
linked above.

There is guaranteed to be a valid solution for each input.

\section*{Scoring}
Your solution will be tested on 5 test cases, each worth a maximum of 20 points.
Your score for each test case will be $20$ times the proportion of characters
of the grid that you have filled in with \texttt{a-z}.
The filled-in letters from the input are also included in this count.

If the grid of characters \texttt{a-z} is not connected, your score for that test case \emph{will be 30\% lower}.

For example, if the sample would have been a real test case, the sample solution
would score $19 / 80 \cdot 20 = 4.75$ points, since it filled in $19$ squares out of $80$.
Because the grid is connected, it does not suffer the 30\% penalty.

Your final score will be the maximum score of a single submission.
