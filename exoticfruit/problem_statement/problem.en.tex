\problemname{Exotic Fruit}
Nicole wants to get into the market of exotic fruit.
She is planning to sell up to $N$ fruits this year, but she hasn't yet decided which kinds.
She has seeds for $K$ different kinds, which she can plant any number of fruits of.

However, planting fruits uses space, water, fertilizer, etc., and thus costs money.
Also, the exotic fruit market has diminishing returns on investment.
A basic requirement for a fruit to be ``exotic'' is that it is rare -- in fact, the
resale price for a fruit of type $i$ is $c_i / a_i$, where $c_i$ is a constant
(dependent on e.g. how cool the fruit looks and how it tastes), and $a_i$ is the number
of fruits of that type on the market. (If $a_i = 0$, the price is undefined.)

For the fruit type $i$ we are also given the numbers $b_i$ -- the cost of planting one
such fruit, and $d_i$ -- the number of such fruits already on the market.
Thus, the win Nicole makes by planting $x$ plants of type $i$ is

\[ x c_i / (d_i + x) - x b_i. \]

How much money can Nicole make on exotic fruit, if she can plant up to $N$ fruits?

\section*{Input}
The first line of input contains the two positive integers $N, K$ ($1 \le N \le 10^{15}, 1 \le K \le 10^5$) -- the maximum number of fruits
Nicole wants to sell this year, and the number of kinds of fruits she has seeds for.

The next $K$ lines of input described the fruits, and each contain three integers $b_i, c_i, d_i$ ($0 \le b_i, c_i, d_i \le 10^9$).

\section*{Output}
Output a single line -- the maximum amount of money Nicole can make by planting up to $N$ seeds.
The answer should have an absolute or relative error of at most $10^{-6}$.

\section*{Scoring}
Your solution will be tested on a set of test groups, each worth a number of points.
To get the points for a test group you need to solve all test cases in the test group.
Your final score will be the maximum score of a single submission.

\noindent
\begin{tabular}{| l | l | l | l |}
\hline
Group & Points & Constraints \\ \hline
1     & 10     & All $d_i = 0$. \\ \hline
2     & 15     & $K = 1$ \\ \hline
3     & 15     & $N \le 30\,000$ \\ \hline
4     & 25     & No additional constraints. \\ \hline
\end{tabular}
